\documentclass{article}
\usepackage{amssymb}

\title{How to use \texttt{DMTesterHAC.jar}}
\date{\today}
\author{Kilho Shin}

\newcommand\dm{\texttt{.dm}}

\begin{document}
\maketitle
\paragraph{Overview.}
Given \dm\ files, 
this program performs
the hierarchical agglomerative clustering (HAC) algorithm
and compares the distance metrics
with respect to the Adjusted Rand Index (ARI) and
the Normalized Mutual Information (NMI). 

\paragraph{Previous program.}

A program that computes distance matrices of trees and
outputs the computed matrices as \dm\ files,
for example, \texttt{TED.jar}.

\paragraph{Next program.}

\texttt{ResultAnalyzer.jar},
which statistically compare the distance metrics
by Wilcoxon Signed Rank Test, Friedman Test, Hommel Test,
Shaffer Test and Benjamini Hachberg Test.

\paragraph{Usage.}
\begin{quotation}\tt
  java -jar DMTesterHAC.jar -c config.txt -w out.result
\end{quotation}

\begin{description}
\item[\tt -c.] (Mandatory) 
  This option specifies a configuration file to use for comparison.
\item[\tt -w:] (Mndatory)
  This option specifies a \texttt{.result} file into which
  the results of comparison will be written.
\end{description} 
\paragraph{Configuration files.}
A configuration looks as follows.
\begin{quotation}\tt
  \noindent
  FILES\% colon-cancer.tree-fttf-0.10.dm\\
  colon-cancer.tree-fttf-0.50.dm colon-cancer.tree-fttf-1.00.dm colon-cancer.tree-fttf-2.00.dm colon-cancer.tree-fttt-1.00.dm\\
  K\_MAX\% 20\\
  LINK\% complete
\end{quotation}

\begin{description}
\item[\tt FILES:] (Mandatory)
  This specifies one or more \dm\ files to be compared.
  The \dm\ files should be outputs of
  a program that computes distance matrices of trees, 
  for example, the \texttt{TED.jar} program.
\item[\tt K\_MAX:] (Optional)
  If a value $n$ is specified,
  the specified distance metrics are compared
  with respect to the cluster number $k$ with $k = 2, \dots, n$.
  If left out, the value of two will be used.
\item[\tt LINK:] (Optional)
  A linkage to be used when the program runs HAC
  is specified.
  The value should be one of
  \texttt{single}, \texttt{complete}, \texttt{upgma}, \texttt{upgmc}, 
  \texttt{wpgma}, \texttt{wpgmc} and \texttt{wand}.
  If left out or a value other than the above is specified,
  the value \texttt{complete} will be used.
\end{description}

\paragraph{Output.}
An output file follows the \texttt{.result} format and as a result
can be processed by the \texttt{ResultAnalyzer.jar} program.

Also, the output includes the following at bottom.
\begin{itemize}
\item
  \LaTeX\ codes to describe the results of comparison as tables.
\item
  Python codes to draw graphs of the results of comparison.
\end{itemize}
\end{document}
